\chapter{Problem definition}
The primary goal of this work is to research how to efficiently work with large amounts of multimedia data in distributed systems. 
In order to guarantee the devised aproach would make sense

\subsection{Use cases}

One of the simplest use cases in which this system might be used is trivial \textit{duplicate detection} of video files.
One might think that a simple checksum of the video files would suffice, but imagine a system like \textit{YouTube} where hours of video content are uploaded 
each second. It's easy to imagine two people uploading the same trailer for an upcoming movie during the same day. In the easiest case, a simple checksum
check would suffice, but what if the videos where encoded using different codecs or even were uploaded in different resolutions?

To further represent how difficult it may be to answer the question of ,,Are those two files the same material?'' let's go back to the example of users
uploading video content to our YouTube-like service. What if the users knew the uploaded material is copy righted and should not be uploaded to the public
internet? As one might imagine, this might happen quite often on a service like YouTube. Let's assume our users know about our ,,duplicate'' detection
algorithm and that we might be partnering with movie companies, so that we have reference material on the servers we can compare the uploaded videos to.
Obviously the users will try to trick our algorithm into not recognising that the uploaded content is the same as our reference material. One of the tricks
often seen on YouTube.com is that the users upload the material ,,mirrored'' which further complicates our matching algorithm -- we must now also be 
resiliant aganist data modified especialy for the goal of us not being able to detect it.

Another, less copyright focused, goal of the presented system is to be able to extensively mine data from the uploaded video content.
Here the canonical example would be a ,,\textit{TOP 10 Movies of All Time}'' video, which obviously contains video material from at least 10 movies,
usualy in the order of 10th, 9th ... until the 1st (best) movie of all time. If we would be able to match parts of each video to their corresponding 
reference materials, we would be able to get meta data about the now recognised movies and even mine out the data what is the best / worst movie of all time,
even without it being written per se - only by looking at the frames in the video. This idea only scratches the surface of what the system implemented during
this thesis work might do, but it will be our test case that we'll be working towards during this paper.

The system implemented during this work represents a basic effort to tackle the above problems as well as these goa



\chapter{Preparing the cluster}

Hadoop's filesystem must be formated before put into use. This is achieved by issuing the \verb|-format| command to the namenode:

\begin{verbatim}
kmalawski@oculus-master > hadoop namenode -format
\end{verbatim}

It is worth pointing out that a "format" takes place only on the namenode, it does not actually touch the datab stored on the datanodes,
but instead it deleted the data stored on the namenode. The Namenode, as explained previously, stores all metadata about where a file is located,
thus, cleaning it's data makes the files stores in HDFS un-usable, since we don't know "where a file's chunks are stored".