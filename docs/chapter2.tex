\chapter{Technologies}
As the core of this work will focus on analysing and benchmarking usage of popular distributed system stacks, it is only fair to begin with introducing the selected components from which the system consist.

This chapter should be treated as a brief introduction into the selected technologies, as very detailed explanations and and implementation details will be provided throughout chapters \ref{chap:system-design} through \ref{chap:perf-scalability}.

% ------------------------------------------------------------------------------------------------------------------------------------------------
\section{Apache Hadoop}
\label{sec:hadoop}
As the system will require the storage of many gigabytes (hundreds) of data the heart of the system will be an appropriate Distributed File System designed to handle such requirements. 

Apache's Hadoop is an umbrella project which includes projects such the \textit{Hadoop Distributed File System} (or \textit{HDFS} for short) which satisfies the requirement of easily enabling space replication factors in distributed environments. Another project related to Hadoop is the ,,Map Reduce'' framework based on the Google white paper \cite{map-reduce} from 2004, yet the re-implementation was first taken up by Yahoo and then open sourced -- becoming what came to be known as \textit{Hadoop}.

Both HDFS as well as the Map Reduce framework will be used predominantly in this work.

% ------------------------------------------------------------------------------------------------------------------------------------------------
\section{Apache HBase}
\label{sec:hbase}

HBase is a column-oriented database \cite{columnar-database} designed by following the Google white paper on their ,,BigTable'' datastore published in 2007.
Column oriented storage of data, as opposed to row oriented (as most SQL databases), as huge advantages when many aggregations over only given columns are performed.

It was selected for this project because it's excellent random-access to data as well as being perfectly suited for sourcing Map Reduce tasks.
HBase stores it Tables on the Hadoop Distributed File System, thus it scales similarly to it, which will be proven in a later section in Chapter \ref{chap:perf-scalability}.

% ------------------------------------------------------------------------------------------------------------------------------------------------
\section{Scala}
\label{sec:scala}
Scala is a functional \textit{and} object-oriented programming language designed by Martin Odersky \cite{scala} running on the Java Virtual Machine.
I selected it as primary implementation language for this project (though other languages used include: ANSI C, Ruby and Bash) because of the compelling 
libraries for building distributed systems using it, such as \textit{Akka} and \textit{Scalding} (introduced in the sections \ref{sec:akka} and \ref{sec:scalding}).

It's functional nature (making it similar to languages such as Lisp or Haskell) is very helpful when performing transform / aggregate operations over collections of data. It should be also noted that Hadoop itself was inspired by languages such as this, because the canonical names of the functions 
performing data transformation and aggregation in functional languages are: ,,map'' and ,,reduce''.

% ------------------------------------------------------------------------------------------------------------------------------------------------
\section{Akka}
\label{sec:akka}

Akka is a library providing an Actor Model \cite{actor-model} based concurrency for Scala (and Java) applications. 
This model may be familiar to some as it has gained popularity thanks to Erlang \cite{erlang} which implements the same concepts.

For the sake of this thesis, Akka has been used both in local (in-jvm) parallel execution as well as remote (across nodes) message passing mode,
in order to balance the workload generated by actors across the entire cluster. This is explained in detail in Chapter \todo{where?}.

% ------------------------------------------------------------------------------------------------------------------------------------------------
\section{Cascading \& Scalding}
\label{sec:chef}
Cascading is a framework built on top Apache Hadoop and enables map reduce authors to think in terms of high level abstractions, such as data ,,flows'' 
and job ,,pipelines'' (series of Map Reduce jobs executed in parallel or sequentially) which have been used extensively in this project.

During the work on this thesis several I have pushed several contributions to the Scalding open source project.
\todo{link contributions}

\todo{link to patches included}
\todo{explain what scalding and cascading are}

% ------------------------------------------------------------------------------------------------------------------------------------------------
\section{phash}
\label{sec:phash}
PHash is short for ,,Perceptual Hash'' and is a sort of hashing algorithm (primarily aimed for use with images), which retains enough information to be 
comparable with another has, yielding ,,how similar'' these hashes are. The details and implementation of it have been explained by Christoph Zauner's \cite{phash}.

This algorithm is used by the system to perform initial similarity analysis between images. 
The algorithm is publicly available, including sources (in C), and may be used in non-commercial applications.

As the goal of this thesis is not introducing such algorithm, but focusing on image analysis in distributed systems,
I decided to use the provided implementation and focus on clustering and scaling problems.
\todo{is this needed?}

% ------------------------------------------------------------------------------------------------------------------------------------------------
\section{Chef}
\label{sec:chef}
Because of how tedious setting up Hadoop clusters is I have early on during the implementation phase of the project decided that cluster 
provisioning and configuration must be fully automated. Opscode's Chef is a tool which enables preparing provisioning scripts in a very readable
way and apply them to a given set of machines.

Using it as well as cloud providers such as Amazon's EC2, Google's ComputeEngine I was able to fully automate a cluster's deployment.

\todo{can i notice ebay?}

Details about the implementation of these recipes are featured in Appendix A.


% ------------------------------------------------------------------------------------------------------------------------------------------------
\section{Other tools and technologies used}
\label{sec:other tools}

\subsection{youtube-dl}
Youtube-dl is a small library written in python and freely available under an Open Source license. 
It was used in order to make downloading source video files from youtube more efficient, as it is aware of multiple available video formats (high / low quality), and offers multiple options useful yet hard to implement for this project -- including for example ,,preferring to download open source video formats'', which allowed me to avoid installing proprietary video codecs on the servers.

\subsection{tesseract-ocr}

Tesseract \cite{tesseract} is a text recognition library developed by Google and freely available to use (including language stems for most popular languages).
This tool has been used in order to extract text from analysed images, providing even more data.

\todo{more?}